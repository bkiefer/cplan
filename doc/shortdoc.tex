\documentclass[11pt,a4paper]{article}

\usepackage{a4wide}
\usepackage{tikz}
\usetikzlibrary{calc}
\newcommand{\cd}[1]{\texttt{#1}}

\begin{titlepage}
\title{\Large \bf Content Planner Module for Talking Robots}
\date{Saarbr\"ucken, \today}
\author{Bernd Kiefer, DFKI Saarbr\"ucken}
\end{titlepage}

\begin{document}
\maketitle
\setlength{\parindent}{0em}

%\pgfdeclareimage[interpolate=true,height=7cm]{GUIWindow}{GUIWindow}
\pgfdeclareimage[interpolate=true,height=7cm]{MainWindow}
{CPlanMainWindow}
\pgfdeclareimage[interpolate=true,width=\textwidth]{AllWindows}
{CPlanAllWindows}
\pgfdeclareimage[interpolate=true,width=.9\textwidth]{TraceWindow}
{CPlanTraceWindow}

\section{Installation}

Extract the given \texttt{cplanner-x.tar} file. Everything unpacked is below
the top-level directory \texttt{utteranceplanner}. That directory contains
three submodules, a generic visualization component (\texttt{dataviz}), an
Emacs-Java connection module (\texttt{j2emacs}) and the planning module
(\texttt{uttplanner}).

The directory \texttt{utteranceplanner/uttplanner} contains a startup script
(\texttt{updebugger}) and also a \texttt{build.xml} file to recompile the
sources from scratch. We will call this directory the \emph{application
  directory} in the following text. Using the build file will also affect the
\texttt{dataviz} subproject the interactive planner is using. The relevant
targets are \texttt{clean}, \texttt{cleanall}, and \texttt{compile}.

To use the Emacs features, Mac users have to use AquaMacs. Plain X11 Emacs
does not seem to provide a server mode.

It is also recommended to compile the Emacs lisp files in \texttt{utteranceplanner/uttplanner/etc}. Either invoke
\texttt{`make'} in this directory, or compile it with your favorite Emacs
incarnation manually (e.g., with AquaMacs on the Mac).

\section{Execution}

The utterance planner is started with the following command:
\begin{verbatim}
updebugger [-e<macs> [<path-to-emacs-binary>]] [-g|-b <inputfile>]
           [-t <traceflags>] <projectfile>
\end{verbatim}

\texttt{projectfile} is a file that contains settings, like the path to a CCG
grammar, and a list of all rule files that should be loaded. The structure of
the project file is described in more detail in section \ref{sec:projectfile}.

By default, the planner reads the project file, and the rule files specified
there, and continues in an interactive console mode with rudimentary readline
support. The switches \texttt{-g} and \texttt{-b} let the planner run in other
modes, namely:

\begin{itemize}
\item batch mode, (\texttt{-b}) processing a set of logical forms that are
  contained in the file that is given as argument to this flag
\item GUI interactive mode (\texttt{-g} command line switch)
\end{itemize}

\noindent{}The \texttt{-t} flag controls tracing onto the console, using a
bitmask for different phases, bit 1 to trace the match phase and bit 2 to trace
the application phase. Thus, \texttt{-t 3} will show trace information for the
matching as well as the applicaton phase. The \texttt{-t} flag can be given
together with \texttt{-g}.  In this case, tracing to the console will be used
only when the \emph{process} button is used, and not when the GUI tracing mode
itself is activated.

In combination with the \texttt{-g} flag, The \texttt{-e} flag connects the GUI
to a newly started Emacs client that will show the result of rule loading,
making files and errors mouse sensitive to facilitate the editing and debugging
of rule files. In Unix/Linux there is usually no need to specify the path to
the Emacs binary. This is different on Macs, because only AquaMacs works as
emacs server, the path to the AquaMacs binary must be specified. If the Emacs
connection is properly set up, the GUI trace window described below allows to
jump directly to the definition of rules used during processing.

\section{Project File}
\label{sec:projectfile}

The project file is a plain text file that is divided into different sections.
The sections are separated by lines specifying the section names, which are
enclosed in square brackets, i.e., \texttt{[Settings]} for the section
containing the project settings. Empty lines are skipped, and everything
following a hash character (\texttt{`\#'}) until the end of line is treated as
comment and ignored.

\begin{figure}[htbp]
\centering\framebox{\hspace*{2em}%
\begin{minipage}{.8\textwidth}\raggedright\ttfamily\vspace*{2ex}
\# relative paths are relative to the location of this file\\[2ex]

[Settings]\\
history\_file = history.txt\\
ccg\_grammar = ../../resources/grammars/openccg/grammar.xml\\[2ex]

[Rules Stage 1]\\
attributes.trf\\
discoursemarkers.trf\\
infostruct.trf\\
transitivity.trf\\[2ex]

[Rules Stage 2]\\
callfun.trf\\
test.trf\\
foo.bar
\vspace*{3ex}
\end{minipage}}
  \caption{Sample Project File}
  \label{fig:projectfile}
\end{figure}

Currently, the content planner distinguishes two kinds of sections, the
\texttt{Settings} section and the \texttt{Rules} section, of which there may be
more than one.

The \texttt{Settings} section contains key-value pairs that are separated by
the equal sign (\texttt{`='}). Currently, there is only one setting in effect
namely the path to the corresponding CCG grammar, whose key must be
\texttt{ccg\_grammar}.

The section names for rules have to start with the word \texttt{Rules}, the
rest before the closing bracket is taken as the stage name for the rule
list that is following the section separator.

In a rule section, every non-empty line must contain the path to one rule
file. Paths can be given either as absolute paths, or relative to the location
where the project file is located. For a description in what order the rules
are processed and what can be done with different stages, see section
\ref{sec:procregime}.

%\par\vspace*{2ex}

\section{GUI Mode}

Figure \ref{fig:allguiwindows} shows all windows opened by the GUI in one
picture. The topmost window is the GUI main window, which is disabled because
the trace window below it is present and acts as a modal window for this
processor. In the bottom left corner is the Emacs window that is connected
to the GUI, with the rule loading output in the lower half and a rule file
for editing in the upper half.

\begin{figure}[htbp]
  \centering
  \pgfuseimage{AllWindows}
  \caption{Content Planner Windows}
  \label{fig:allguiwindows}
\end{figure}

When started in GUI mode, the application looks for the file \texttt{.cplanner}
in the application directory. This file has the same syntax as the project
file described above. It also has a \texttt{Settings} section to set two of the
the command line arguments permanently, so that they do not have to be provided
on every startup, and a \texttt{RecentFiles} section, which is maintained by
the application, that contains the recently opened project files.

Figure \ref{fig:preferencesfile} shows a sample \texttt{.cplanner} file.

\begin{figure}[htbp]
  \centering
  \centering\framebox{\hspace*{2em}%
\begin{minipage}{.8\textwidth}\raggedright\ttfamily\vspace*{2ex}
\# relative paths are relative to the location of this file\\[2ex]

[Settings]\\
tracing = none \# none / match / action / all\\
emacs  = /home/kiefer/Applications/AquaMacs/bin/Aquamacs\\[2ex]

[RecentFiles]\\
Rules/allrules.cpj\\
\vspace*{3ex}
\end{minipage}}

  \caption{Sample Preferences File}
  \label{fig:preferencesfile}
\end{figure}


\subsection{Main Window}

The main window, shown in figure \ref{fig:mainwindow}, consist of the input
area at the top, and the input and output LF displays below it, and a status
bar for status and error messages at the bottom. The function of the buttons
in the tool bar and the menu items will be described in what follows.

\begin{figure}[htbp]
  \centering
  \begin{tikzpicture}[-stealth, thick,
    every node/.style={font=\sffamily}]
    \node (0,0) [anchor=center] {\pgfuseimage{MainWindow}};
    %\draw[help lines] (-5,-5) grid (5,5); \draw[fill] (0,0) circle (2pt);
    \draw (-4.5, 3.7) node[anchor=south] {Action Buttons} -- (-5, 2.6);
    \draw (0, 3.7) node[anchor=south] {Input Area} -- (0.8, 1.8);
    \draw (-4.3, -3.7) node[anchor=north] {Status Bar} -- (-4, -3.4);
    \draw (-1, -3.7) node[anchor=north] {Input (as Graph)} -- (-2, -1.6);
    \draw (2.5, -3.7) node[anchor=north] {Output Graph} -- (3, -3);
  \end{tikzpicture}
  %\pgfuseimage{GUIWindow}
  \caption{Content Planner Main Window}
  \label{fig:mainwindow}
\end{figure}

\subsubsection{File Menu}
When running in GUI Mode, it is not necessary (put possible) to specify the
project file on the command line. The file can also be opened using the
\texttt{File > Open} menu item. \texttt{New} will open a completely fresh
planner with empty rule set, which means that a rule file must be opened from
the new window for the planner to work. The \texttt{File > Recent Files}
submenu will contain the last project files that have been opened.

\subsubsection{Button Toolbar}
The button toolbar is located below the menu bar, and contains buttons for, in
this sequence:
\begin{itemize}\addtolength{\itemsep}{-.33\itemsep}
\item Reloading the rule set (if it has been changed in an editor)
\item Processing the logical form in the input area
\item Starting processing of the logical form in tracing mode, which opens a
  new modal trace window that is described below
\item Emergency Stop in case the process is in an infinite loop
\item Wipe the input area clean
\item Running the CCG realizer for the last output LF, provided a CCG
  grammar was given on startup
\end{itemize}

If the input contains ill-formed input, and the \textit{Process} or
\textit{Start Trace} button has been pressed, the input area gets red and the
caret will be located at the point where the error seemingly occured. In
addition, the status line (at the bottom of the window) contains an error
message describing the error.

\subsection{Trace Window}

Figure \ref{fig:tracewindow} shows the trace window that will be opened when
the trace button in the main window is pressed. It allows to follow the
processing of the content planner in single or larger steps, and even going
backwards and forwards using the scroll bar at the bottom of the window.

\begin{figure}[htbp]
  \centering
  \pgfuseimage{TraceWindow}
  \caption{Trace Window}
  \label{fig:tracewindow}
\end{figure}

The trace window consists of two main sections, the \emph{match} section, which
is in the upper half and consists of the LF window in the upper left, the
bindings window in the upper right and the text window below these. This text
area shows the matching part of the rule that is currently `processed' and
has been matched successfully against the structure in the window above it.

Since the processor first executes all rule matches pseudo-parallel to the
structure and after that applies the actions one by one, this window will
change not as often as the windows in the action section (see also section
\ref{sec:procregime}).

The second section of the trace window are the two LF windows in the bottom half
and the text area for the action part of the rule below them. The lower left LF
window shows the current state of the dag, before the action(s) that are shown
in the text area are applied, while the lower right LF window shows the new dag
after application of the action(s).

If an Emacs connection has been established, double-clicking into the match
or action text area will open the appropriate rule file and jump directly to
the location of the rule whose application is currently shown.

Additionally, there is a button toolbar at the top of the window, which allows
you to perform the following actions:

\begin{itemize}\addtolength{\itemsep}{-.33\itemsep}
\item Jump to the current rule in the editor (the same as double clicking into
  the text areas)
\item Suspend processing for very long or infinite runs
\item Continue suspended processing
\item Emergency stop (finishes processing)
\item Close the trace window
\end{itemize}

\newpage
\section{General Description}
The new utterance planner module is very much like a specialized graph
rewriting system for logical forms as they are used in OpenCCG. A set of
rules is specified by the grammar writer which is applied to every part of
an input logical form, which is interpreted as a directed graph where the
nominals and feature values are the nodes and the features and relations are
the (named) edges.

The transfer of nodes is realized using local and global variables, which
store information either during the application of a single rule, or during
the whole rewrite process, respectively.

\section{Rules}

Rules have the following form:
{
\newcommand{\nt}[1]{\emph{#1}}
%\newcommand{\tok}[1]{\texttt{'#1'}}
\newcommand{\tok}[1]{\texttt{\textbf{#1}}}
\newcommand{\id}{\mbox{\texttt{ID}}}
\begin{center}

$\underbrace{\mbox{\nt{match} ( \tok{,}
    (\nt{global\_variable} \textbar\ \nt{function}\tok{(}\nt{args}*\tok{)})
    \tok{\^} \nt{match} )*}}_{\mbox{matching part}} \mbox{\ \tok{->}\ }
\underbrace{\mbox{\nt{action} ( \tok{,} \nt{action} )*}}_{\mbox{action
    part}} \mbox{\ \tok{.}}$

\end{center}

\subsection{Matching Part}
\label{matching_part}
The reduced BNF for one match looks like this:

\begin{flushleft}
\nt{match} $\rightarrow$
  \nt{pathmatch} (\tok{\^{}} \textbar\  \tok{|} ) \nt{match} \textbar\
  \nt{pathmatch}\\
\nt{pathmatch} $\rightarrow$
  \tok{<} ID \tok{>} [ \nt{pathmatch} ] \textbar\
  \nt{simplematch}\\
\nt{simplematch} $\rightarrow$
  \nt{atomic} \tok{:} \textbar\
  \nt{atomic} [ \tok{:} \nt{atomic} ] \textbar\
  \tok{!} \nt{pathmatch} \textbar\
  \tok{(} \nt{match} \tok{)} \textbar\ \tok{=} \nt{var} \tok{:} \\
\nt{atomic} $\rightarrow$
  \mbox{\texttt{ID}} \textbar\
  \tok{\#} \mbox{\texttt{ID}}\\
\nt{var} $\rightarrow$
  \tok{\#} \mbox{\texttt{ID}} \textbar\
  \tok{\#\#} \mbox{\texttt{ID}}\\
\end{flushleft}
}
The match part consists of atomic match expressions which can be combined with
conjunction, disjunction and negation operators to obtain complex match
expressions:

\begin{itemize}
\item \cd{nom:} matches the nominal with name \cd{nom}
\item \cd{:type1} matches if the node has type \cd{type1}
\item \cd{prop} matches either the proposition value \cd{prop}, or the value
  under some feature
\item \cd{<fname>} matches if \cd{fname} either exists as feature or
  relation
\end{itemize}

Additionally, \emph{local variables} also form atomic match expressions. They
can bind either whole nodes, type or proposition values, and can also be used
several times in a match expression to check that the value of the node is
(structurally) equal to what is bound to the variable. Local variables start
with a single hash (\cd{\#}), followed by a valid name. The syntax is as
follows:

\begin{itemize}
\item \cd{\#v:} matches a whole node (nominal).
\item \cd{:\#v} matches the type value of a nominal.
\item \cd{\#v} matches the proposition value, or the value under a feature.
\item \cd{= \#var:} checks if the node that is matched is \emph{identical} to
  what is bound to the variable \emph{var}. An unbound variable will always
  generate a match failure for this test.
\end{itemize}

Alternatively, a (bound) global variable can be used for the above checks.
Global variables can not be bound on the match side, but their values can
be checked against the local value as for local variables.

%\newpage
There are two binary and one unary combination operator to combine atomic
matches into more complex match expressions:

\begin{itemize}\addtolength{\itemsep}{-.5\itemsep}
\item[\textbf{\cd{\^}}] for conjunction
\item[\textbf{\cd{|}}] for disjunction
\item[\textbf{\cd{!}}] for negation
\end{itemize}

Parentheses can be used to group subexpressions.
%\newpage

There are also some short forms of conjunctions following the logical form
syntax that do not require the conjunction operator, such as:

\begin{itemize}
\item[] \cd{<feat>(nom:\#type2)}
\end{itemize}
matches a node that has a relation \cd{feat} which points to a node with
nominal name \cd{nom} and binds the type of that nominal to the local variable
\cd{\#type2}.

The first match, which may be empty, will always be applied to the current node
that is under consideration. For a detailed description of how the matching an
modification of the dag works, see section \ref{sec:procregime}. The subsequent
match conditions, which are separated by commas, match against global variables
that have to be specified at the beginning of the match condition. Global
variables, in contrast to local variables, must begin with two hash characters,
e.g., \texttt{\#\#global\_var}. If the global variable is not bound, the match
is executed as if it were applied to an independent, empty node.

It is also possible to call functions in the matching part, as additional
matches, similar to the matching against a global variable. Those functions
may have either atomic values or (already bound) local variables as arguments,
and the result value will be matched in same way as against the content of
a global variable, or the current structure.

The following extensions are planned for the matching side:
\begin{itemize}
\item Global and local variables to match against (part) of the current node,
  either checking equality or coreference
\item type subsumption if a CCG type system is available (should that be an
  operator different from \cd{\^} ?)
\end{itemize}

\subsection{Actions}

The BNF for one action looks like this:
{
\newcommand{\nt}[1]{\emph{#1}}
%\newcommand{\tok}[1]{\texttt{'#1'}}
\newcommand{\tok}[1]{\texttt{\textbf{#1}}}
\newcommand{\id}{\mbox{\textit{\textsf{ID}}}}
\begin{flushleft}
\nt{action} $\rightarrow$ \nt{appoint} \tok{!} \tok{<}\id \tok{>} \textbar\
\nt{appoint} \tok{=} \nt{rexpr} \textbar\  \nt{appoint} \tok{\^{}} \nt{rexpr}\\
\nt{rexpr} $\rightarrow$
  \nt{rpathexpr} \tok{\^{}} \nt{rexpr} \textbar\
  \nt{rpathexpr}\\
\nt{rpathexpr} $\rightarrow$
  \tok{<}\id \tok{>} \nt{rpathexpr} \textbar\
  \nt{rsimpleexpr}\\
\nt{rsimpleexpr} $\rightarrow$
  \nt{atomic} \tok{:} [ \nt{atomic} ] \textbar\
  \nt{atomic} \textbar\ \ \verb$"STRING"$  \textbar\
  \tok{(} \nt{rexpr} \tok{)} \textbar\
  \id\tok{(}\nt{args}\tok{)}\\
\nt{atomic} $\rightarrow$ \id \ \textbar\ \nt{var}\\
\nt{var} $\rightarrow$
  \tok{\#}\id \ \textbar
  \ \tok{\#\#}\id [ \tok{<}\id \tok{>} ( \tok{,} \tok{<}\id \tok{>} )* ] \textbar\
  \ \tok{\#\#\#}\id \\
\nt{appoint} $\rightarrow$ \tok{\#}  \ \textbar\ \nt{var}\\
\nt{args} $\rightarrow$ $\epsilon$
  \textbar\ \nt{atomic} (\tok{,} \nt{atomic})*
\end{flushleft}
}

\newpage
An action consists of the following elements:
\begin{itemize}
\item first the specification of the place it is applied to
\item an optional feature path
\item an operator specifying the kind of action that is
  to be performed (addition, replacement, deletion)
\item the structure that specifies what is added, replaced or deleted.
\end{itemize}

The place where to apply the action can be either a local or global variable,
or a single hash character which represents the node against which the rule is
matched.

For addition and replacement, full or partial logical forms, also containing
local variables, can be specified. Deletion can currently only be applied to
single features or relations.

Minimal examples for the three operations are:
\begin{list}{}{\setlength{\leftmargin}{3.2cm}\setlength{\labelwidth}{2.6cm}
\renewcommand{\makelabel}[1]{\textbf{#1}}}
\item[Addition] \cd{\# \^\  <feat> val}\\ Add a feature / value pair to the current
  node
\item[Replacement] \cd{\#var = <feat> \#var}\\ this would place the nominal that
  was bound to the local variable \cd{var} in the matching part under the
  feature \cd{feat} instead of its former location.
\item[Deletion] \cd{\# !\ <p3>}\\ delete feature \cd{p3} from the node that is
  currently processed
\end{list}

Assignment and replacement of global variables can also be used to inhibit
rules, or force a certain ordering on the application of rules (see the
section about matching, especially against the content of global variables).
In addition, a global variable can be followed by a nonempty path of features,
providing a map-like structure. On the match side, these substructures are
treated like features of structured nodes and have to be extracted for binding
or testing with the usual conjunction and path syntax.

In addition to the local and global variables, \emph{right hand side local
  variables} can be used in multi-part action specifications. They start
with three hash characters and their purpose is three fold:
\begin{itemize}
\item provide a means of specifying complex replacements in several steps for
  clarity
\item create temporary structured nodes as arguments to functions.
\item establish coreferences in complex nodes on the right hand side, because
  you are not allowed to use unbound local variables
\end{itemize}

If variables are used in the replacement of additions and assignments, no
matter which kind of variables that is, you have to keep the following in mind:

\begin{itemize}
\item if you use \texttt{\#var:}, the complete node bound to the variable will
  be added to the specified point of application (which can be an embedded
  path)
\item if you use \texttt{:\#var} or \texttt{\#var} or \texttt{<feat>\#var}, the
  value bound to the variable must be of a special form, because you use the
  content of var to specify an atomic value:
  \begin{enumerate}
  \item Either it must be a simple (i.e., atomic) value, i.e., a variable bound
    to a proposition, type, or feature value, \emph{or}
  \item It contains a complex node that has a feature with the same name as
    specified in the replacement, that is, a type, proposition or \texttt{feat}
    feature, respectively
  \end{enumerate}
  If the above conditions are not met, an error will be signalled.
\end{itemize}

This does not mean that you can assign a type value only to a type, you can
move the atomic values around, if you do it the right way.  So the following
will always work (you take the type and put it under feature \texttt{x}):
\begin{verbatim}
:#var -> # ^ <x> #var.
\end{verbatim}

This will work only if the node bound to \texttt{var} has a valid type and
proposition:
\begin{verbatim}
#var: -> # ^ <y> ( :#var ^ #var )
\end{verbatim}

\subsection{Some Simple Examples}

Test for a disjunction of types, and the existence of the \cd{Shape} feature,
add a new nominal with the feature's value as proposition and delete the
\cd{Shape} feature itself, to avoid infinite recursion. Since the same
relation can be added multiple times, the rewriting process would never reach
a fix point and, consequently, not stop.

\begin{verbatim}
(:entity | :physical | :e-substance) ^ <Shape> #v
->
# ^ <Modifier> (shape:q-shape ^ #v),
# ! <Shape>.
\end{verbatim}

A similar example, but now, the whole node under \cd{TopIn} is moved under the
\cd{Anchor} feature, which specifies with the parentheses and the nominal
specification with the colon after the variable that it is a relation.
Note that under \cd{Modifier}, no specification of the nominal name is
necessary, which results in a new, unique nominal name for that node.

\begin{verbatim}
(:entity | :physical | :e-substance) ^ <TopIn> ( #topin: )
->
# ^ <Modifier> (:m-location ^ in ^ <Anchor> ( #topin: )),
# ! <TopIn>.
\end{verbatim}

In the next example, a local variable is used only in the action part multiple
times to indicate the coreferencing of two nominals. The whole content under
\cd{Content} will be replaced due to the replacement specification for the
matched local variable \cd{c1}

\begin{verbatim}
:dvp ^ <Relation> accept ^ <Content> ( #c1:marker )
->
#c1 = (:ascription ^ <Actor>(#i1:person ^ I)
                   ^ <Patient>(:entity ^ what)
                   ^ <Subject>( #i1: )).
\end{verbatim}
%\newpage

The next example adds the result of a function call to the current node. The
function gets as argument the structure that is bound to the variable
\texttt{c1}. A single \texttt{\#} is also a valid argument to a function and
represents the node that is currently processed, like in the initial place of
the action specification.

\begin{verbatim}
:dvp ^ <Relation> accept ^ <Content> ( #c1:marker )
->
# ^ <Content> modify_marker(#c1).
\end{verbatim}

Functions can also be used on the matching side (see section
\ref{matching_part}), as in the next example:

\begin{verbatim}
:dvp, random(one, two, three) ^ two
->
# = <Content> ( <Answer> positive ).
\end{verbatim}

Also look into the file \texttt{presentation-examples.trf}, which contains
much more elaborate examples with explanations attached to it.

\section{Adding New Functions}

New functions can be added using a lightweight plug-in mechanism. All functions
must be subclasses of \texttt{Function}, if they work on the internal
structures of the planner directly, or of \texttt{LFFunction}, if they work
with \texttt{LogicalForm}, as provided with the separate \texttt{CPlanWrapper}
class. The methods that must be overwritten for both classes are:

\begin{verbatim}
public abstract String name();
public abstract int arity();
\end{verbatim}

\texttt{name} must return the name as it is used in the rules, and arity the
expected length of the argument list.

Additionally, the abstract method \verb|Object apply(List args)| must be
implemented for every subclass of \texttt{Function}, except for
\texttt{LFFunction}, where \verb|LogicalForm applyLfFunction(List args)|
must be implemented instead. The length of args matches the arity specified by
the function.

If the functions are not directly compiled in and registered at the
\texttt{FuncctionFactory}, they can be provided as plug-ins. The class files
of the functions have to be added to a jar-archive such that the path where
they are stored in the archive matches the package prefix.

\begin{figure}[htbp]
\begin{verbatim}
$ jar tf plugin-jars/cplan-plugins.jar

META-INF/
META-INF/MANIFEST.MF
de/dfki/lt/tr/dialogue/cplanwrapper/ChangePropFunction.class
\end{verbatim}
\caption{\texttt{de.dfki.lt.tr.dialogue.cplanwrapper.ChangePropFunction}
  properly put into a \texttt{.jar} file to use as plug-in function }
\end{figure}

These jars have to be put into one directory, which is passed to the
constructor of the \texttt{UtterancePlanner} class as second argument. The
constructor will then load all appropriate classes (subclasses of
\texttt{Function}) on startup.


\section{Processing Regime\label{sec:procregime}}

Processing is divided into different stages. Every stage has its own set of
rules, which are loaded from the rule files specified in different sections of
the project file (see \ref{sec:projectfile} for further information), and
essentially behaves in the same way.

For every stage, the current processing engine proceeds as follows:
\begin{itemize}
\item Traverse the structure in preorder sequence, starting at the root node
\item At every node, match all rules against the node in the order in which
  they have been loaded

  \noindent{}If the match was successful:\\[-3.4ex]
  \begin{itemize}
  \item[] store the rule together with the
    local bindings of the match for later execution of the rule's actions
  \end{itemize}
\item Execute the actions for the stored rule / binding pairs in the order in
  which they were stored, which is the order in which the matches occured.
\end{itemize}

Since the matches are applied before any changes are made, any applicable match
will be found. When the changes are applied afterwards, is is possible that
some of them will not have any effect on the input structure, say, if something
is added to a node that has previously been deleted.

This process is iterated until a fixpoint is reached, i.e., the structure does
not change by applying the rules as described. If the current stage was not the
last stage, the structure is then processed with the next set of rules, until
all stages have been applied. Every stage is only applied once, there is no
fixpoint loop around the stages. The stages are applied in the order in which
they appear in the project file, independent of their names.

\if0
\section{Future Enhancements}
\begin{itemize}
\item Macros to avoid copy and paste duplication of parts of rules
\item GUI enhancements
  \begin{itemize}
  \item Input history, with the possibility to get/write in from/to file
  \item Save last rule file location/rule file name and possibly other settings
    to a customization file.
  \item redirecting console output to a separate window which can be shown or
    hidden on demand.
  \end{itemize}
\end{itemize}
\fi

\end{document}
