\example{1}: Sub-node is replaced with fresh content. Notice the match
condition for the absence of a feature.

\begin{verbatim}
/// @d:dvp(<SA>assertion ^ <Rel> accept ^ <Cont> (:conttype ^ <foo> bar))
:dvp ^ ! <Ackno> ^ <SA> assertion ^ <Rel> accept ^ <Cont> ( #c1: )
->
#c1 = (:marker ^ ok).
\end{verbatim}

\example{2}: Selecting a specific relation, and adding to it
\begin{verbatim}
/// @aa:bb(<C>(<Mod>(:g ^ <x> y)))
<C>(<Mod>(#m:g ^ ! <Cont>)) -> #m ^ <Cont>(:new ^ clean).
\end{verbatim}

\example{3}: Add a default in case of feature absence
\begin{verbatim}
/// @aa:ascr(<C>d)
:ascr ^ !<Tense> -> # ^ <Tense>pres.
\end{verbatim}

\example{4}: Set a global variable as marker, and test in the second rule
\begin{verbatim}
/// @d:dvp(<SpeechAct>assertion ^ <w>(:foo ^ <Tense>pres))
:dvp ^ <SpeechAct>#v -> ##speechact = #v.

:foo ^ <Tense> ^ (##speechact ~ assertion) -> # ^ <Mood>ind.
\end{verbatim}


\example{5}: type disjunction, variable \texttt{t} matching the whole node
add two relations, delete \texttt{Target} feature (multiple actions)
\begin{verbatim}
/// @d:disj(<T>(:entity ^ <Tense>pres))
/// @d:disj(<T>(:thing ^ <Tense>pres))
:disj ^ <T> #t:(entity | thing)
->
# ^ <CR>(#t:) ^ <Subject>(#t:),
# ! <T>.
\end{verbatim}

\example{6}: Less preferable rewrite of the last example, the same variable
name has to be used twice! Works only because of disjunction.
\begin{verbatim}
/// @d:disj2(<T>(:entity ^ <Tense>pres))
/// @d:disj2(<T>(:thing ^ <Tense>pres))
:disj2 ^ (<T> (#t:entity) | <T> (#t:thing))
->
# ^ <CR>(#t:), # ^ <Subject>(#t:),
# ! <T>.
\end{verbatim}

\example{7}: Randomizing with complex values
\begin{verbatim}
/// @d:rand(<SpeechAct>opening ^ <Content>(:top ^ <X> y))
:rand ^ <SpeechAct>opening ^ <Content> (#c1:top)
->
###opening1 = :opening ^ "hi, dude",
###opening2 = :opening ^ hello,
###opening3 = :opening ^ "nice to see you" ^ <form> polite,
# ! <SpeechAct>,
/// Note the colon after the function call! It means that the whole node is the
/// value, not just the proposition.
#c1 = random(###opening1, ###opening2, ###opening3): .

/// @d:rand2(<SpeechAct>opening ^ <Content>(:top ^ <X> y))
:rand2 ^ <SpeechAct>opening ^ <Content> (#c1:top)
->
###opening1 = "hi, dude",
###opening2 = hello,
###opening3 = "nice to see you" ^ <form> polite,
# ! <SpeechAct>,
#c1 = random(###opening1, ###opening2, ###opening3):opening .

\end{verbatim}

\example{8}: Alternative randomization, maybe not very convenient
\begin{verbatim}
/// @d:dvp(<SpeechAct>closing ^ <Content> (foo))
:dvp ^ <SpeechAct>closing -> ##randomclosing = random(1,2).

:dvp ^ <SpeechAct>closing ^ <Content> (#c1:) ^
(##randomclosing ~ 1)
->
#c1 = :closing ^ bye.

:dvp ^ <SpeechAct>closing ^ <Content> (#c1:) ^
(##randomclosing ~ 2)
->
#c1 = :closing ^ see_you.
\end{verbatim}

\example{9}: Using global variable as node store.

After application of these rules \texttt{Target} and \texttt{PointToTarget}
point to the same node.

\begin{verbatim}
/// @d:dvp(<Speechact>assertion ^ <Content>(a:ascription))
:ascription ^ #t: -> ##fromStore = #t:.

:ascription ^ ! <PointToTarget>
->
Again, note the colon after the global variable in the addition
# ^ <PointToTarget> ##fromStore:.
\end{verbatim}


\subsection{
 Adding to relations the wrong way
}

\begin{verbatim}
/* Test input
@a:dvp(foo ^ <SpeechAct>question ^
             <Content>(c1:ascription ^
                 <Subject>(s1:entity ^ <Delimitation>unique) ^
                 <Cop-Scope>(s2:gaga ^ prop ^ <Questioned>true)))
 */
\end{verbatim}
\example{10}
This will not add to the existing \texttt{Subject} and \texttt{Cop-Scope},
but introduce new ones.
\begin{verbatim}
:dvp ^ <SpeechAct>question
     ^ <Content>(#cont:ascription ^
                 <Subject>(:entity ^ <Delimitation>unique) ^
                 <Cop-Scope>(#cop-scope: ^ <Questioned>true))
->
#cont ^ <Wh-Restr>(:specifier ^ what ^ <Scope> #cop-scope:)
      ^ <Subject>( context ^ <Proximity> proximal )
      ^ <Cop-Scope>(<Delimitation>unique ^ <Num> sg),
#cop-scope ! <Questioned>.
\end{verbatim}

Adding to relations: the correct alternative:\\
This adds the new information to the previously matched nodes.
\begin{verbatim}
:dvp ^ <SpeechAct>question
     ^ <Content>(#cont:ascription ^
                 <Subject>(#subj:entity ^ <Delimitation>unique) ^
                 <Cop-Scope>(#cop-scope: ^ <Questioned>true))
->
#cont ^ <Wh-Restr>(:specifier ^ what ^ <Scope> #cop-scope:),
#subj ^ context ^ <Proximity> proximal,
#cop-scope ! <Questioned>,
#cop-scope ^ <Delimitation>unique ^ <Num> sg.
\end{verbatim}
\newpage
\subsection{
 Global variable ``maps''
}
\example{11} Global variables can be used with path expressions to get
dictionary-like structures
\begin{verbatim}
/// @d:dvp2(<Cont> (x:y ^ z) ^ <S>x)
:dvp2 ^ <Cont> #c: ^ <S>#s: -> ##gvar<cont> = #c:, ##gvar<s><t> = #s:.

:dvp2 ^ !<SpeechAct> ^ !<Cont2> ^ ( ##gvar ~ <cont> #v: )
->
# ^ <Cont2> ( #v: ^ <S> ##gvar<s><t>: ).

/// or, alternatively for the second rule (uncomment to test)
/*
:dvp2 ^ !<SpeechAct> ^ !<Cont2> ^ ( ##gvar ~ <cont> )
->
# ^ <Cont2> ( ##gvar<cont>: ^ <S> ##gvar<s><t>:).
*/
\end{verbatim}

\subsection{
 All you can do with variables
}
\example{12} Bind values to global variables and use them in another rule
\begin{verbatim}
/// @d:dvp(<foo>(:a ^ <F>(b:c ^ d)))
:a ^ <F> (#i:#t ^ #p) -> ##partial = :#t ^ #p, ##whole = #i:.

:a ^ !<W> -> # ^ <W> ##whole: ^ <P> ##partial:.
\end{verbatim}


 \example{13} Be careful that you use bound variables correctly! If you use
 them as simple (type or proposition) values on the right hand side, you
 must have bound them to simple values, or complex values containing the
 appropriate edge!

The second rule illustrates how to get type and prop out of a complex node
in a global variable
\begin{verbatim}
/// @d:test(<Actor>(:type ^ prop ^ <foo> bar))
<Actor>(#a:) -> ##s = #a:.

:test ^ ! <Subject> ^ (##s ~ (:#type ^ #prop))
->
# ^ <Subject> ##s: ^ <Prop> #prop ^ <Type> :#type.
\end{verbatim}

Right hand local variables can also be used to establish coreferences in the
replacement part. \vspace*{1.0ex}

\example{14} Check for structural equality (will also succeed if coreferent)
\begin{verbatim}
/// @d:dvp(<Content>(:bar ^ baz) ^ <Wh-Restr>(:bar ^ baz))
:dvp ^ <Content>#c: ^ <Wh-Restr>#c: -> # ^ :equals.
\end{verbatim}

\example{15} Check for identity (will only succeed if coreferent)
\begin{verbatim}
/// @d:coref(<Content>(:bar ^ baz) ^ <Wh-Restr>(:bar ^ baz))
/// @d:coref(<Content>(a:bar ^ baz) ^ <Wh-Restr>(a:bar))
/// @d:coref(<Content>(a:bar ^ baz) ^ <Wh-Restr>a:)
:coref ^ <Content>#c: ^ <Wh-Restr> = #c: -> # ^ identical.
\end{verbatim}

\subsection{
 Use of functions for tests and results
}
\example{16} Bind values to global variables and use them in another rule
\begin{verbatim}
/// @d:dvp(<SpeechAct>provideQuestion ^ <Context>(<Question> "question " ^ <Count> "1" ))
:dvp ^ <SpeechAct>provideQuestion
     ^ <Context>(<Question> #q ^ <Count> #x: ) ^
( eq(#x, 1) ~ 1 )
->
###part1 = random("La prima domanda �: ",
                  "Ecco la prima domanda: ",
                  "Qui viene la prima domanda: ",
                  "Puoi rispondere a la prima domanda: "),
# = :canned ^ <string>concatenate(###part1, #q, "?").
\end{verbatim}

\example{17} Non-integer numbers must be passed to functions as strings. Boolean functions
return zero or one.
\begin{verbatim}
/// @m:math(<arg1>9 ^ <arg2>2)
/// @m:math(<arg1>9 ^ <arg2>"30.33")

:math ^ <arg1>#arg1 ^ <arg2>#arg2 ^ (lteq("0.3", div(#arg1, #arg2)) ~ 1)
->
# ^ <res> div(#arg1, #arg2).

:math ^ <arg1>#arg1 ^ <arg2>#arg2
->
# ^ <bool> lteq("0.3", div(#arg1, #arg2)).
\end{verbatim}

\example{18} You can use arbitrary encodings in your grammar files,
but if it's not UTF-8, you have to specify as `encoding' setting in the
grammar file, and, consequently, you can only have one encoding per project.
\begin{verbatim}
/// @e:enc(<enc> iso-8859-15 ^ <val> "�����")
<enc> iso-8859-15 ^ <val> "�����" -> # ^ <right> true.
\end{verbatim}

\example{19} Using \tok{=>} rules to walk through a list:
\begin{verbatim}
///@d:dvp(<c>0 ^ <l>(:li ^ <car>1 ^ <cdr>(:li ^ <car>2 ^ <cdr>(:li ^ <car>3 ^ <cdr>()))))
///@d:dvp(<c>0123 ^ <l>(:li))

<c>#c: ^ <l>(#l: ^ <car>#f ^ <cdr>#r:)
=>
#l = #r:,
#c = concatenate(#c, #f).
\end{verbatim}

%%% Local Variables:
%%% mode: latex
%%% TeX-master: "shortdoc"
%%% End:
